\documentclass[11pt, a4paper, twocolumn]{IEEEtran}

\title{ars}
\author{KWB ENTERPRISES LTD}
\date{\today}
\usepackage{algorithmic}
\usepackage{color}
% Default fixed font does not support bold face
\DeclareFixedFont{\ttb}{T1}{txtt}{bx}{n}{12} % for bold
\DeclareFixedFont{\ttm}{T1}{txtt}{m}{n}{12} % for normal
\definecolor{deepblue}{rgb}{0,0,0.5}
\definecolor{deepgreen}{rgb}{0,0.5,0}
\begin{document}
\maketitle
\section{WELCOME TO ROME}
AS OVID SAID \cite{ArsArmatoria}: 

Love is like warfare … The night, winter, long marches, cruel suffering, painful toil, all these things have to be borne by those who fight in Love's campaigns ... If the ordinary, safe route to your mistress is denied you, if her door is shut against you, climb up on to the roof and let yourself down by the chimney, or the skylight. How it will please her to know the risks you've run for her sake! 'Twill be an earnest of your love.” 
\section{some Heidegger}
A lot more interesting it is to analyze how it is to be \cite{SeinZeit}:
 Das »Sein« ist der »allgemeinste« Begriff: tÕ Ôn œsti kaqÒlou
m£lista p£ntwn.1 Illud quod primo cadit sub apprehensione est
ens, cuius intellectus includitur in omnibus, quaecumque quis
apprehendit. »Ein Verständnis des Seins ist je schon mit Inbegriffen
in allem, was einer am Seienden erfaßt.«2 Aber die
»Allgemeinheit« von »Sein« ist nicht die der Gattung. »Sein«
umgrenzt nicht die oberste Region des Seienden, sofern dieses
nach Gattung und Art begrifflich artikuliert ist: oÜte tÕ Ôn g◊noj.3
Die »Allgemeinheit« des Seins »übersteigt« alle gattungsmäßige
Allgemeinheit. »Sein« ist nach der Bezeichnung der mittelalterlichen
Ontologie ein »transcendens«. Die Einheit dieses transzendental
»Allgemeinen« gegenüber der Mannigfaltigkeit der sachhaltigen
obersten Gattungsbegriffe hat schon Aristoteles als die
Einheit der Analogie erkannt. Mit dieser Entdeckung hat Aristoteles
bei aller Abhängigkeit von der ontologischen Fragestellung
Platons das Problem des Seins auf eine grundsätzlich neue Basis
gestellt. Gelichtet hat das Dunkel dieser kategorialen Zusammenhänge
freilich auch er nicht. Die mittelalterliche Ontologie hat
dieses Problem vor allem in den thomistischen und skotistischen
Schulrichtungen vielfältig diskutiert, ohne zu einer grundsätzlichen
Klarheit zu kommen. Und wenn schließlich Hegel das
»Sein« bestimmt als das »unbestimmte Unmittelbare« und diese
Bestimmung allen weiteren kategorialen Explikationen seiner
»Logik« zugrunde legt, so hält er sich in derselben Blickrichtung
wie die antike Ontologie, nur daß er das von Aristoteles schon
gestellte Problem der Einheit des Seins gegenüber der Mannigfaltigkeit
der sachhaltigen »Kategorien« aus der Hand gibt. Wenn
man demnach sagt: »Sein« ist der allgemeinste Begriff, so kann
das nicht heißen, er ist der klarste und aller weiteren Erörterung
unbedürftig. Der Begriff des »Seins« ist vielmehr der dunkelste.
\section{CODE}
now some code. Because everything without code is boring.
\begin{algorithmic}
    \STATE $S=0$
\IF{i == 2} \STATE{j = 3;} \ELSE \STATE{j=2;} \ENDIF
\WHILE{true} \STATE{System.out.println("You're pretty dumb!")} \ENDWHILE
\PRINT YOU can't be intelligent.
\end{algorithmic}
\section{some more random code}
\lstset{
language=Java,
basicstyle=\ttm,
keywordstyle=\ttb\color{{deepblue}},
stringstyle=\color{deepgreen},
frame=tb,
showstringspaces=false
}
\begin{lstlisting}

public class HelloWorld
\{
	public static void main(String[] args)\{ 
		System.out.println("Hello World!");
	\}
\}
\end{lstlisting}
\lstinputlisting{Ex1.java}
\bibliographystyle{ieeetr}
\bibliography{bibliography} % note that .bib is not included.
\end{document}

